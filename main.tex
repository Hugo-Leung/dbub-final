\documentclass[reprint,aps,unsortedaddress,superscriptaddress,prl,floatfix,showpacs,linenumbers]{revtex4-2}
\usepackage[utf8]{inputenc}
\usepackage{float}
\usepackage{graphicx}
\usepackage{amsmath,amsthm,amssymb}
\usepackage{verbatim}
\usepackage{url}
\usepackage{subcaption}
\usepackage[separate-uncertainty=true]{siunitx}
\usepackage{physics}
\usepackage{hyperref}
\usepackage[capitalize]{cleveref}
\usepackage[obeyFinal]{todonotes}
\setuptodonotes{inline}
\usepackage{adjustbox}
\usepackage{multirow}
\usepackage{lineno}
\linenumbers
\graphicspath{{figures/}}
\DeclareSIUnit\barn{b}

\captionsetup{justification=raggedright,singlelinecheck=false}

\begin{document}

\title{Improve measurement of flavor asymmetry of the light-quark sea in the proton with Drell-Yan production in
	\texorpdfstring{$p+p$}{p+p} and \texorpdfstring{$p+d$}{p+d} collisions at \texorpdfstring{\SI{120}{\GeV}}{120~GeV}}
% !TeX root = main.tex
\author{C.H.~Leung\,\orcidlink{0000-0001-7907-3728}}
\email[Contact author: ]{cleung@jlab.org}
\altaffiliation[Present address: ]{Thomas Jefferson National Accelerator Facility, Newport News, Virginia 23606, USA}
\affiliation{Department of Physics, University of Illinois at Urbana-Champaign, Urbana, Illinois 61801, USA}

\author{J.~Arrington\,\orcidlink{0000-0002-0702-1328}}
\altaffiliation[Present address: ]{Lawrence Berkeley National Laboratory, Berkeley, California, 94720 USA}
\affiliation{Physics Division, Argonne National Laboratory, Lemont, Illinois 60439, USA}

\author{W.C.~Chang\,\orcidlink{0000-0002-1695-7830}}
\affiliation{Institute of Physics, Academia Sinica, Taipei,11529, Taiwan}

\author{D.C.~Christian\,\orcidlink{0000-0003-1275-6510}}
\affiliation{Fermi National Accelerator Laboratory, Batavia, Illinois 60510, USA}

\author{J.~Dove}
\affiliation{Department of Physics, University of Illinois at Urbana-Champaign, Urbana, Illinois 61801, USA}

\author{D.F.~Geesaman\,\orcidlink{0000-0003-2557-3131}}
\affiliation{Physics Division, Argonne National Laboratory, Lemont, Illinois 60439, USA}

\author{R.S.~Guo}
\affiliation{Department of Physics, National Kaohsiung Normal University, Kaohsiung 824, Taiwan}

\author{M.F.~Hossain\,\orcidlink{0000-0002-6467-1394}}
\affiliation{Department of Physics, New Mexico State University, Las Cruces, NM 88003 USA}
\affiliation{University of Virginia, Charlottesville, VA, 22904 USA}

\author{E.R.~Kinney\,\orcidlink{0000-0002-4176-5283}}
\affiliation{Department of Physics, University of Colorado, Boulder, Colorado 80309, USA}

\author{D.~Kleinjan\,\orcidlink{0000-0002-2737-0859}}
\affiliation{Los Alamos National Laboratory, Los Alamos, New Mexico 87545, US}

\author{C.D.~Kuruppu\,\orcidlink{0000-0003-2772-1978}}
\affiliation{Department of Physics, New Mexico State University, Las Cruces, NM 88003 USA}

\author{K.~Liu\,\orcidlink{0000-0002-6676-8165}}
\affiliation{Los Alamos National Laboratory, Los Alamos, New Mexico 87545, US}

\author{W.~Lorenzon\,\orcidlink{0000-0003-0657-8463}}
\affiliation{Randall Laboratory of Physics, University of Michigan, Ann Arbor, Michigan 48109, USA}

\author{S.~Miyasaka\,\orcidlink{0009-0004-1293-5679}}
\affiliation{Department of Physics, Tokyo Institute of Technology, Meguro-ku, Tokyo 152-8550, Japan}

\author{K.~Nagai\,\orcidlink{0000-0002-5336-8306}}
\altaffiliation[Present address: ]{Department of Physics and Materials Science, The University of Memphis, Memphis, TN 38152}
\affiliation{Los Alamos National Laboratory, Los Alamos, New Mexico 87545, US}
\affiliation{Institute of Physics, Academia Sinica, Taipei,11529, Taiwan}
\affiliation{Department of Physics, Tokyo Institute of Technology, Meguro-ku, Tokyo 152-8550, Japan}

\author{J.C.~Peng\,\orcidlink{0000-0003-4198-9030}}
\affiliation{Department of Physics, University of Illinois at Urbana-Champaign, Urbana, Illinois 61801, USA}

\author{S.~Prasad\,\orcidlink{0000-0003-3404-0062}}
\affiliation{Department of Physics, University of Illinois at Urbana-Champaign, Urbana, Illinois 61801, USA}
\affiliation{Physics Division, Argonne National Laboratory, Lemont, Illinois 60439, USA}

\author{P.E.~Reimer\,\orcidlink{0000-0002-0301-2176}}
\affiliation{Physics Division, Argonne National Laboratory, Lemont, Illinois 60439, USA}

\author{F.~Sanftl}
\affiliation{Department of Physics, Tokyo Institute of Technology, Meguro-ku, Tokyo 152-8550, Japan}

\author{S.~Sawada}
\affiliation{Institute of Particle and Nuclear Studies, KEK, High Energy Accelerator Research Organization, Tsukuba, Ibaraki 305-0801, Japan}

\author{T.~Sawada\,\orcidlink{0000-0001-5726-7150}}
\affiliation{Randall Laboratory of Physics, University of Michigan, Ann Arbor, Michigan 48109, USA}

\author{T.-A.~Shibata\,\orcidlink{0009-0005-5498-4804}}
\altaffiliation[Present address: ]{Nihon University, College of Science and Technology, Chiyoda-ku, Tokyo 101-8308, Japan}
\affiliation{Department of Physics, Tokyo Institute of Technology, Meguro-ku, Tokyo 152-8550, Japan}
\affiliation{RIKEN Nishina Center for Accelerator-Based Science, Wako, Saitama 351-0198, Japan}

\author{R.~Towell\,\orcidlink{0000-0003-3640-7008}}
\affiliation{Department of Engineering and Physics, Abilene Christian University, Abilene, Texas 79699 USA}

\author{S.~Uemura\,\orcidlink{0000-0003-3458-4625}}
\altaffiliation[Present address: ]{Fermi National Accelerator Laboratory, Batavia, Illinois 60510, USA}
\affiliation{Los Alamos National Laboratory, Los Alamos, New Mexico 87545, US}

\author{S.G.~Wang\,\orcidlink{0000-0001-8474-9817}}
\altaffiliation[Present address: ]{APS, Argonne National Laboratory, Lemont, Illinois 60439, USA}
\affiliation{Institute of Physics, Academia Sinica, Taipei,11529, Taiwan}
\affiliation{Department of Physics, National Kaohsiung Normal University, Kaohsiung 824, Taiwan}
\affiliation{Fermi National Accelerator Laboratory, Batavia, Illinois 60510, USA}

\author{J.~Wu\,\orcidlink{0000-0003-4432-9521}}
\affiliation{Fermi National Accelerator Laboratory, Batavia, Illinois 60510, USA}

\author{N.~Wuerfel\,\orcidlink{0000-0001-9872-5330}}
\affiliation{Randall Laboratory of Physics, University of Michigan, Ann Arbor, Michigan 48109, USA}

\author{Z.H.~Ye\,\orcidlink{0000-0002-1873-2344}}
\altaffiliation[Present address: ]{Department of Physics, Tsinghua University, Beijing 100084, China}
\affiliation{Physics Division, Argonne National Laboratory, Lemont, Illinois 60439, USA}

\collaboration{FNAL E906/SeaQuest Collaboration}


\date{\today}

\begin{abstract}
	We report improved results from the SeaQuest Fermilab E906 experiment with improved statistical precision for
	$\bar{d}\left(x\right) / \bar{u}\left(x\right)$ in the large $x$ region up to $x=0.45$ using the 120 GeV proton beam.
	The $\bar{d}\left(x\right) / \bar{u}\left(x\right)$ ratios and the $\bar{d}\left(x\right) - \bar{u}\left(x\right)$
	differences are deduced from these cross section ratios for $0.13 < x < 0.45$.
	The new results on $\bar{d}\left(x\right) / \bar{u}\left(x\right)$ and $\bar{d}\left(x\right) - \bar{u}\left(x\right)$
	are compared with various parton distribution functions and theoretical calculations.
\end{abstract}


\maketitle

\section{Introduction}
\todo{are we comparing with E866 here?}
\todo{summarized previos finding?}
\todo{discribe the results from STAR and why this new result is better}

The results based on Run 2-3 have been previously reported in Refs.~\cite{dove2021,dove2023}.
Several recent global PDF analyses~\cite{cocuzza2021,ball2022a,accardi2023,alekhin2023}
have included these results, in addition to the $W$ boson production data from the STAR collaboration~\cite{adam2021}.
In this letter, we report the improved results using all data collected by the SeaQuest Experiment at Fermilab.
The data constitute a two-fold increase in the detected muons compared to our previous publications.
\todo{also summarized any improvement in the analysis if any}

\section{The Drell-Yan Process}
\label{sec:drell-yan}
The SeaQuest experiment detects $\mu^+\mu^-$ pairs (dimuons) produced in the interaction of a proton
beam with various targets nuclei. The production of massive dimuon pairs are described by the Drell-Yan
process~\cite{drell1970}. In the Drell-Yan process, the leading order (LO) cross section is given by
\begin{multline}
	\frac{d^2\sigma}{dx_1dx_2}=\frac{4\pi \alpha^2}{9x_1x_2s} \times
	\label{eq:DYCross} \\
	\sum_{i\in u,d,s,\dots} e_i^2 \left[q_i^A\left(x_1\right) \bar q_i^B\left(x_2\right) + \bar q_i^A\left(x_1\right)
		q_i^B\left(x_2\right)\right],
\end{multline}
where $\alpha$ is the fine-structure constant, $e_i$ is the charge of a quark with flavor $i$,
and $q_i^{A,B}\left(x_{1,2}\right)$ are the quark distribution functions in hadrons $A$ and $B$
for quarks carrying a momentum fraction $x_1$ and $x_2$, respectively.
An analogous notation is used for antiquark distribution functions $\bar q_i^{A,B}\left(x_{1,2}\right)$.

The experiment measures the momenta of the $\mu^+$ and $\mu^-$, $p_{\mu^+}$ and $p_{\mu^-}$.
From these, the four-momentum $Q$ of virtual photon from the quark-antiquark annihilation is determined,
$Q = p_{\mu^+} + p_{\mu^-}$.
The variables $x_1$ and $x_2$ of the quark-antiquark pair are then
\begin{equation}
	x_1 = \frac{P_2 \cdot Q}{P_2 \cdot P} \text{ and } x_2 =
	\frac{P_1 \cdot Q}{P_1 \cdot P},
	\label{def_x1x2}
\end{equation}
where $P_1$ and $P_2$ are the four-momenta of the projectile and target hadron,
respectively, and $P$ is the sum of $P_1$ and $P_2$, $P=P_1+P_2$.
The invariant mass-squared of the dimuon, $M^2 = Q^2$, is related to $x_1$, $x_2$, $s$, and $P_T$ by
\begin{equation}
	M^2=x_1 x_2 s - P_T^2,
	\label{def_mass}
\end{equation}
where $P_T$ is the transverse momentum of the dimuon.

The Feynman-$x$, $x_F$, of the dimuon is
\begin{equation}
	x_F = \frac{P_L}{P_{\text{max}}} = \frac{2P_L}{\sqrt{s}\left(1-M^2/s\right)},
	\label{def_xf}
\end{equation}
where $P_L$ is the longitudinal momentum of the dimuon and $P_\textrm{max}$ is the maximum momentum in the center-of-mass frame of the colliding hadrons.
With the definition of $x_F$ adopted in \cref{def_xf},
the full range of $\left|x_F\right| < 1$ is covered.
In contrast, for the alternative definition of $x_F = 2P_L/\sqrt{s}$,
frequently used in the literature,
the coverage of $x_F$ is limited to $\left|x_F\right| <\left(1-M^2/s\right)$.
These two definitions of $x_F$ converge at the high-energy limit when $M^2/s \to 0$.
It is worth noting that \cref{def_mass} becomes the familiar expression $M^2=x_1 x_2 s$,
often adopted in the literature, when $P_T/\sqrt{s} \to 0$.
For the SeaQuest experiment, the modest beam momentum of \SI{120}{\GeV} $\left(\sqrt{s} = \SI{15.1}{\GeV}\right)$
warrants the use of the definitions of kinematic variables shown in \cref{def_mass} and \ref{def_xf}.

The leading order Drell-Yan cross section, expressed in \cref{eq:DYCross},
contains two terms since the annihilation could proceed with the antiquark from either parent hadron.
For most fixed-target experiments, including SeaQuest,
the spectrometers have large acceptance for the positive $x_F$  region $\left(x_F > 0\right)$.
Thus, to a good approximation, the Drell-Yan cross section is dominated by the first term,
corresponding to the annihilation of a beam quark with a target antiquark.
To demonstrate the relation between the cross section ratio and the flavor asymmetry
$\bar{d}\left(x\right) / \bar{u}\left(x\right)$,
an approximate formula can be derived as follows.
Taking into account the dominance of $u\left(x\right)$ over $d\left(x\right)$ in
beam proton and the charge squared weighting factor $e_i^2$ in \cref{eq:DYCross}, one obtains for $x_1 \gg x_2$,
\begin{align}
	\frac{\sigma^{pd}}{2\sigma^{pp}} = \frac{1}{2}\left[1 + \frac{\sigma^{pn}}{\sigma^{pp}}\right]
	\approx \frac{1}{2}\left[1 + \frac{\bar{u}_n\left(x_2\right)}{\bar{u}_p\left(x_2\right)}\right],
\end{align}
with the assumption that $\sigma^{pd} \approx \sigma^{pp} + \sigma^{pn}$,
which neglects small nuclear effects of the deuteron~\cite{kumano1998,ehlers2014}.
Charge symmetry for the parton distributions~\cite{londergan2010} between the proton and the neutron:
\begin{equation}
	\begin{split}
		\bar u\left(x\right)  \equiv & \bar u_p\left(x\right) = \bar d_n\left(x\right)\textrm{, and } \\
		\bar d\left(x\right)  \equiv & \bar d_p\left(x\right) = \bar u_n\left(x\right),
	\end{split}
	\label{eq:chargeSymmetry}
\end{equation}
leads to the approximate expression
\begin{equation}
	\frac{\sigma^{pd}}{2\sigma^{pp}} \approx
	\frac{1}{2} \left[1+\frac{\bar d\left(x_2\right)}{\bar u\left(x_2\right)}\right].
	\label{eq:crRatio}
\end{equation}
While \cref{eq:crRatio} illustrates the power of the $\sigma^{pd}/2\sigma^{pp}$ Drell-Yan cross section ratio to reveal the flavor asymmetry between $\bar d$ and $\bar u$, the actual extraction of the $\bar d\left(x\right) / \bar u\left(x\right)$ ratios from the measured $\sigma^{pd}/ 2 \sigma^{pp}$ Drell-Yan cross section ratios is made without these simplifying approximations and is carried out in Next-to-Leading Order (NLO) in the strong coupling constant, $\alpha_s$, as discussed in Sec.~\ref{sec:extraction}.

\section{The SeaQuest Experiment}
The SeaQuest experiment receives the  proton beam from the Fermilab Main Injector at \SI{120}{\GeV}
once every minute in 4-second periods (spills).
The proton beam is further grouped into \SI{2}{\ns} ``buckets'' every \SI{18.8}{\ns},
inherited from the accelerator \SI{53.1}{\MHz} radiofrequency (RF) structure.
The average intensity of the beam is approximately $2\times 10^{12}$ protons per second.
The proton beam in the Main Injector are peeled off using a process known as resonance
extraction over the course of the \SI{4}{\second} spill.
While the average instantaneous intensity was roughly $10^4$ protons per buckets,
the measured numbers varied up to few times of $10^5$ protons.

Buckets with a large number of protons can produce large numbers of hits in the detector,
easily satisfy the trigger requirements, and can saturate the data  
Moreover, these events are typically too noisy to be reconstructed effectively.
\subsection{Target System}
\subsection{Spectrometer}
\begin{figure*}
	\centering
	\includegraphics[width=0.8\linewidth]{spectrometer/twoColumnSeaQuestSpectrometerNIM.pdf}
	\caption{Schematic of the SeaQuest spectrometer.}
	\label{fig:spectrometer}
\end{figure*}
The schematic layout od the SeaQuest spectrometer, discussed in detail in Ref.~\cite{aidala2019},
is displayed in \cref{fig:spectrometer}.

\subsection{Trigger System}

\begin{table}[tb]
	\centering
	\caption{Triggers used in the present analysis. The T or B denotes the top or bottom section traversed by the track. \label{tab:triggers}}
	\begin{tabular}{c@{\hspace{6\tabcolsep}}c@{\hspace{6\tabcolsep}}c@{\hspace{6\tabcolsep}}l}
		\hline
		\hline
		Trigger & Side  & Charge  & Description        \\
		\hline
		1       & TB/BT & $+-/-+$ & Unlike-sign dimuon \\
		4       & T/B   & $+/-$   & Single muon        \\
		\hline
		\hline
	\end{tabular}
\end{table}

\section{Data Analysis}
\todo{progress between nature paper and now?}
\subsection{Track reconstruction}

\subsection{Monte Carlo simulation}
\todo{Explain the event generation, geat4 options, embeding(?), realization(?).}
\todo{How well is our data discribed by GMC? Which distributions are we comparing?}

\subsection{Mass-fit method}
\todo{Explain the method. If we are using NMSU mix, we will need to justify the normalization.}
\begin{figure}[htbp!]
	\centering
	\begin{subfigure}{\linewidth}
		\includegraphics[width=\linewidth]{massfit_run56_LH2.pdf}
	\end{subfigure}
	\begin{subfigure}{\linewidth}
		\includegraphics[width=\linewidth]{massfit_run56_LD2.pdf}
	\end{subfigure}
	\caption{Dimuon mass distribution for events collected
		on liquid hydrogen (top) and deuterium (bottom) targets for the second data set.
		The data points (solid squares) are compared with a fit (solid blue line) consisting of
		various components (see text).}
	\label{fig:massfit}
\end{figure}
\missingfigure[figwidth=\linewidth]{Drell-Yan yeild vs $x_T$ after background subtraction compared with MC}
\todo{corrections(?): rate dependence, target contamination, live PoT}

\section{Measurement of The \texorpdfstring{$\sigma_{pd}/2\sigma_{pp}$}{pd/2pp} Drell-Yan Cross Section Ratio}
\todo{which variables do we want to show? (x2, x1, mass, pT, xF, etc)}
\begin{figure}[htbp!]
	\centering
	\includegraphics[width=\linewidth]{E906_E866_xT_CT18only.pdf}
	\caption{Measured $\sigma_{pd}/2\sigma_{pp}$ Drell-Yan cross section ratio from SeaQuest compared with E866/NuSea~\cite{towell2001}
		and calculations using CT18~\cite{hou2021} at different kinematics.}
	\label{fig:xT_csr}
\end{figure}

\todo{break down of systematics}
\begin{table*}
	\centering
	\caption{Breakdown of systematic uncertainty for $\sigma_{pd}/2\sigma_{pp}$.}
	\begin{adjustbox}{max width=\linewidth}
		\begin{tabular}{c|cccccc}
\hline\hline
$x_2$                      & $0.13-0.16$         & $0.16-0.195$        & $0.195-0.24$        & $0.24-0.29$         & $0.29-0.35$         & $0.35-0.45$         \\ \hline
$\sigma_{pd}/2\sigma_{pp}$ & $1.177\pm0.033$     & $1.174\pm0.022$     & $1.275\pm0.024$     & $1.232\pm0.029$     & $1.205\pm0.040$     & $1.212\pm0.055$     \\ \hline
$\delta_{Eff.\,stat.}$     & \SI{0.57}{\percent} & \SI{0.37}{\percent} & \SI{0.36}{\percent} & \SI{0.41}{\percent} & \SI{0.51}{\percent} & \SI{0.69}{\percent} \\
$\delta_{Eff.\,param.}$    & \SI{0.41}{\percent} & \SI{0.28}{\percent} & \SI{0.46}{\percent} & \SI{0.27}{\percent} & \SI{0.26}{\percent} & \SI{0.27}{\percent} \\
$\delta_{Mix}$             & \SI{1.00}{\percent} & \SI{0.48}{\percent} & \SI{0.41}{\percent} & \SI{2.17}{\percent} & \SI{3.82}{\percent} & \SI{3.23}{\percent} \\
$\delta_{Empty}$           & \SI{0.14}{\percent} & \SI{0.10}{\percent} & \SI{0.16}{\percent} & \SI{0.15}{\percent} & \SI{0.13}{\percent} & \SI{0.15}{\percent} \\
$\delta_{Beam\,norm.}$     & \SI{2}{\percent}    & \SI{2}{\percent}    & \SI{2}{\percent}    & \SI{2}{\percent}    & \SI{2}{\percent}    & \SI{2}{\percent}    \\ \hline
$\delta_{syst.}$           & $\pm 0.028$         & $\pm 0.025$         & $\pm 0.027$         & $\pm 0.037$         & $\pm 0.052$         & $\pm 0.047$         \\ \hline\hline
\end{tabular}

	\end{adjustbox}
\end{table*}
\todo{consistence of the data(source of systematics)}

\section{Extraction of \texorpdfstring{$\bar{d}\left(x\right)/\bar{u}\left(x\right)$}{dbar(x)/ubar(x)}
  and \texorpdfstring{$\bar{d}\left(x\right)-\bar{u}\left(x\right)$}{dbar(x)-ubar(x)}}
\label{sec:extraction}
\todo{Explain the extraction. There are also addition systematics here.}
\todo{Where to put acceptance? We did show the acceptance in nature paper.}
\begin{table*}[htbp!]
	\centering
	\caption{The measured $\sigma_{pd}/2\sigma_{pp}$ cross section ratio as well
		as the extracted $\bar{d}/\bar{u}$ and $\bar{d}-\bar{u}$ for each $x_{2}$ bin.
		The first uncertainty is statistical and the second systematic.}
	\begin{adjustbox}{max width=\textwidth}
		% !TeX root = ../main.tex
{
\renewcommand{\arraystretch}{1.5}
\begin{tabular}{cccccccc}
	\hline\hline
	$x_{2}$ range    & $\expval{x_{2}}$ & $\expval{x_{1}}$ & \begin{tabular}{@{}c@{}} $\expval{p_{T}}$\\(\unit{\GeV/c})\end{tabular} & \begin{tabular}{@{}c@{}}$\expval{M}$\\(\unit{\GeV/c^2})\end{tabular} & $\sigma_{pd}/2\sigma_{pp}$ & $\bar{d}/\bar{u}$                     & $\bar{d}-\bar{u}$                     \\ \hline
	$0.130$--$0.160$ & $0.146$          & $0.687$          & $0.760$                         & $4.71$                        & $1.177\pm0.033\pm0.028$    & $1.383^{+0.058+0.060}_{-0.053-0.060}$ & $0.176^{+0.021+0.024}_{-0.022-0.023}$ \\
	$0.160$--$0.195$ & $0.181$          & $0.610$          & $0.759$                         & $4.87$                        & $1.174\pm0.022\pm0.025$    & $1.431^{+0.041+0.061}_{-0.051-0.061}$ & $0.111^{+0.011+0.011}_{-0.011-0.011}$ \\
	$0.195$--$0.240$ & $0.222$          & $0.553$          & $0.760$                         & $5.11$                        & $1.275\pm0.024\pm0.027$    & $1.672^{+0.052+0.082}_{-0.052-0.082}$ & $0.092^{+0.012+0.012}_{-0.012-0.012}$ \\
	$0.240$--$0.290$ & $0.263$          & $0.516$          & $0.761$                         & $5.44$                        & $1.232\pm0.029\pm0.037$    & $1.653^{+0.073+0.123}_{-0.073-0.113}$ & $0.043^{+0.003+0.013}_{-0.003-0.013}$ \\
	$0.290$--$0.350$ & $0.324$          & $0.492$          & $0.762$                         & $5.83$                        & $1.205\pm0.040\pm0.052$    & $1.694^{+0.124+0.174}_{-0.114-0.174}$ & $0.024^{+0.004+0.004}_{-0.004-0.004}$ \\
	$0.350$--$0.450$ & $0.395$          & $0.474$          & $0.762$                         & $6.34$                        & $1.212\pm0.055\pm0.047$    & $1.925^{+0.205+0.205}_{-0.195-0.205}$ & $0.015^{+0.005+0.005}_{-0.005-0.005}$ \\ \hline\hline
\end{tabular}
}

	\end{adjustbox}
\end{table*}
\begin{table*}[htbp!]
	\centering
	\caption{Values of $\int_{0.45}^{0.13} \left[\bar{d}\left(x\right) - \bar{u}\left(x\right)\right] \dd{x}$
		and $\int_{0.45}^{0.13} x\left[\bar{d}\left(x\right) - \bar{u}\left(x\right)\right] \dd{x}$ at $Q^2=\SI{25.5}{\GeV\squared}$ extracted from
		SeaQuest compared with CT18, NNPDF4.0 PDFs as well as meson cloud and statistical models.}
	\begin{adjustbox}{max width=\textwidth}
		% !TeX root = ../main.tex
\renewcommand{\arraystretch}{1.5}
\begin{tabular}{cccccccc}
\hline \hline
 &          & \multicolumn{3}{c}{PDFs} & & \multicolumn{2}{c}{Models} \\ \cline{3-5} \cline{7-8}
 & SeaQuest & CJ15 &CJ22     & NNPDF4.0     & & Stat.     & Meson cloud    \\ \hline
$\int^{0.45}_{0.13} \left[\bar{d}\left(x\right) - \bar{u}\left(x\right) \right]\dd{x}$ &
  $0.0179_{-0.0018}^{+0.0017} {}_{-0.0023}^{+0.0022}$ &
  $0.0153^{+0.0025}_{-0.0024}$ &
  $0.0167^{+0.0009}_{-0.0028}$ &
  $0.0208^{+0.0036}_{-0.0036}$ & &
  $0.0186$ &
  $0.0180$ \\
$\int^{0.45}_{0.13} x\left[\bar{d}\left(x\right) - \bar{u}\left(x\right) \right]\dd{x}$ &
  $0.00368_{-0.00036}^{+0.00034} {}_{-0.00049}^{+0.00045}$ &
  $0.00271^{+0.00050}_{-0.00046}$ &
  $0.00319^{+0.00019}_{-0.00063}$ &
  $0.00414^{+0.00078}_{-0.00078}$ & &
  $0.00386$ &
  $0.00361$ \\ \hline \hline
\end{tabular}

	\end{adjustbox}
\end{table*}

\begin{figure}[htpb!]
	\centering
	\begin{subfigure}{\linewidth}
		\includegraphics[width=\linewidth]{E906_E866_dbarubar_PDF.pdf}
	\end{subfigure}
	\begin{subfigure}{\linewidth}
		\includegraphics[width=\linewidth]{E906_E866_dbarubar.pdf}
	\end{subfigure}
	\caption{The extracted $\bar{d}/\bar{u}$ from the measured $\sigma_{pd}/2\sigma_{pp}$ Drell-Yan cross section ratio
		from SeaQuest (red solid circles), E866~\cite{towell2001} (open blue circles).
		The extracted ratios are compared with CT18~\cite{hou2021} and NNPDF4.0~\cite{ball2022a} global PDF analyses (top)
		and with predictions from meson cloud model~\cite{alberg2022} and statistical model~\cite{soffer2019} (bottom).}
	\label{fig:e906_e866_dbarubar}
\end{figure}

\begin{figure}[htpb!]
	\centering
	\begin{subfigure}{\linewidth}
		\includegraphics[width=\linewidth]{dbub_diff_with_PDF.pdf}
	\end{subfigure}
	\begin{subfigure}{\linewidth}
		\includegraphics[width=\linewidth]{dbub_diff_with_model.pdf}
	\end{subfigure}
	\caption{The extracted $\bar{d}-\bar{u}$ from from SeaQuest (red solid circles), E866~\cite{towell2001} (open blue circles)
		and HERMES~\cite{ackerstaff1998} (black triangles).
		The results are compared with CT18~\cite{hou2021} and NNPDF4.0~\cite{ball2022a} global PDF analyses (top)
		and with predictions from meson cloud model~\cite{alberg2022} and statistical model~\cite{soffer2019} (bottom).
		The bottom panel is also zoomed in on $x>0.1$ in order to emphasis the SeaQuest results.}
	\label{fig:e906_e866_dbarMubar}
\end{figure}


\section{Impact on PDF analysis}
\todo{Not sure about this part}

\section{Conclusions}

\begin{acknowledgments}
	\todo{Add funding information}
\end{acknowledgments}
\bibliography{reference}

\end{document}

