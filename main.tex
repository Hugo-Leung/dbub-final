\documentclass[reprint,aps,unsortedaddress,superscriptaddress,prd,floatfix,showpacs,linenumbers]{revtex4-2}
\usepackage[utf8]{inputenc}
\usepackage{float}
\usepackage{graphicx}
\usepackage{amsmath,amsthm,amssymb}
\usepackage{verbatim}
\usepackage{url}
\usepackage{subcaption}
\usepackage[separate-uncertainty=true]{siunitx}
\usepackage{physics}
\usepackage{hyperref}
\usepackage[capitalize]{cleveref}
\usepackage[obeyFinal]{todonotes}
\setuptodonotes{inline}
\usepackage{adjustbox}
\usepackage{multirow}
\usepackage{lineno}
\linenumbers
\graphicspath{{figures/}}
\DeclareSIUnit\barn{b}

\captionsetup{justification=raggedright,singlelinecheck=false}

\begin{document}

\title{Improve measurement of flavor asymmetry of the light-quark sea in the proton with Drell-Yan production in
	\texorpdfstring{$p+p$}{p+p} and \texorpdfstring{$p+d$}{p+d} collisions at \texorpdfstring{\SI{120}{\GeV}}{120~GeV}}
% !TeX root = main.tex
\author{C.H.~Leung\,\orcidlink{0000-0001-7907-3728}}
\email[Contact author: ]{cleung@jlab.org}
\altaffiliation[Present address: ]{Thomas Jefferson National Accelerator Facility, Newport News, Virginia 23606, USA}
\affiliation{Department of Physics, University of Illinois at Urbana-Champaign, Urbana, Illinois 61801, USA}

\author{J.~Arrington\,\orcidlink{0000-0002-0702-1328}}
\altaffiliation[Present address: ]{Lawrence Berkeley National Laboratory, Berkeley, California, 94720 USA}
\affiliation{Physics Division, Argonne National Laboratory, Lemont, Illinois 60439, USA}

\author{W.C.~Chang\,\orcidlink{0000-0002-1695-7830}}
\affiliation{Institute of Physics, Academia Sinica, Taipei,11529, Taiwan}

\author{D.C.~Christian\,\orcidlink{0000-0003-1275-6510}}
\affiliation{Fermi National Accelerator Laboratory, Batavia, Illinois 60510, USA}

\author{J.~Dove}
\affiliation{Department of Physics, University of Illinois at Urbana-Champaign, Urbana, Illinois 61801, USA}

\author{D.F.~Geesaman\,\orcidlink{0000-0003-2557-3131}}
\affiliation{Physics Division, Argonne National Laboratory, Lemont, Illinois 60439, USA}

\author{R.S.~Guo}
\affiliation{Department of Physics, National Kaohsiung Normal University, Kaohsiung 824, Taiwan}

\author{M.F.~Hossain\,\orcidlink{0000-0002-6467-1394}}
\affiliation{Department of Physics, New Mexico State University, Las Cruces, NM 88003 USA}
\affiliation{University of Virginia, Charlottesville, VA, 22904 USA}

\author{E.R.~Kinney\,\orcidlink{0000-0002-4176-5283}}
\affiliation{Department of Physics, University of Colorado, Boulder, Colorado 80309, USA}

\author{D.~Kleinjan\,\orcidlink{0000-0002-2737-0859}}
\affiliation{Los Alamos National Laboratory, Los Alamos, New Mexico 87545, US}

\author{C.D.~Kuruppu\,\orcidlink{0000-0003-2772-1978}}
\affiliation{Department of Physics, New Mexico State University, Las Cruces, NM 88003 USA}

\author{K.~Liu\,\orcidlink{0000-0002-6676-8165}}
\affiliation{Los Alamos National Laboratory, Los Alamos, New Mexico 87545, US}

\author{W.~Lorenzon\,\orcidlink{0000-0003-0657-8463}}
\affiliation{Randall Laboratory of Physics, University of Michigan, Ann Arbor, Michigan 48109, USA}

\author{S.~Miyasaka\,\orcidlink{0009-0004-1293-5679}}
\affiliation{Department of Physics, Tokyo Institute of Technology, Meguro-ku, Tokyo 152-8550, Japan}

\author{K.~Nagai\,\orcidlink{0000-0002-5336-8306}}
\altaffiliation[Present address: ]{Department of Physics and Materials Science, The University of Memphis, Memphis, TN 38152}
\affiliation{Los Alamos National Laboratory, Los Alamos, New Mexico 87545, US}
\affiliation{Institute of Physics, Academia Sinica, Taipei,11529, Taiwan}
\affiliation{Department of Physics, Tokyo Institute of Technology, Meguro-ku, Tokyo 152-8550, Japan}

\author{J.C.~Peng\,\orcidlink{0000-0003-4198-9030}}
\affiliation{Department of Physics, University of Illinois at Urbana-Champaign, Urbana, Illinois 61801, USA}

\author{S.~Prasad\,\orcidlink{0000-0003-3404-0062}}
\affiliation{Department of Physics, University of Illinois at Urbana-Champaign, Urbana, Illinois 61801, USA}
\affiliation{Physics Division, Argonne National Laboratory, Lemont, Illinois 60439, USA}

\author{P.E.~Reimer\,\orcidlink{0000-0002-0301-2176}}
\affiliation{Physics Division, Argonne National Laboratory, Lemont, Illinois 60439, USA}

\author{F.~Sanftl}
\affiliation{Department of Physics, Tokyo Institute of Technology, Meguro-ku, Tokyo 152-8550, Japan}

\author{S.~Sawada}
\affiliation{Institute of Particle and Nuclear Studies, KEK, High Energy Accelerator Research Organization, Tsukuba, Ibaraki 305-0801, Japan}

\author{T.~Sawada\,\orcidlink{0000-0001-5726-7150}}
\affiliation{Randall Laboratory of Physics, University of Michigan, Ann Arbor, Michigan 48109, USA}

\author{T.-A.~Shibata\,\orcidlink{0009-0005-5498-4804}}
\altaffiliation[Present address: ]{Nihon University, College of Science and Technology, Chiyoda-ku, Tokyo 101-8308, Japan}
\affiliation{Department of Physics, Tokyo Institute of Technology, Meguro-ku, Tokyo 152-8550, Japan}
\affiliation{RIKEN Nishina Center for Accelerator-Based Science, Wako, Saitama 351-0198, Japan}

\author{R.~Towell\,\orcidlink{0000-0003-3640-7008}}
\affiliation{Department of Engineering and Physics, Abilene Christian University, Abilene, Texas 79699 USA}

\author{S.~Uemura\,\orcidlink{0000-0003-3458-4625}}
\altaffiliation[Present address: ]{Fermi National Accelerator Laboratory, Batavia, Illinois 60510, USA}
\affiliation{Los Alamos National Laboratory, Los Alamos, New Mexico 87545, US}

\author{S.G.~Wang\,\orcidlink{0000-0001-8474-9817}}
\altaffiliation[Present address: ]{APS, Argonne National Laboratory, Lemont, Illinois 60439, USA}
\affiliation{Institute of Physics, Academia Sinica, Taipei,11529, Taiwan}
\affiliation{Department of Physics, National Kaohsiung Normal University, Kaohsiung 824, Taiwan}
\affiliation{Fermi National Accelerator Laboratory, Batavia, Illinois 60510, USA}

\author{J.~Wu\,\orcidlink{0000-0003-4432-9521}}
\affiliation{Fermi National Accelerator Laboratory, Batavia, Illinois 60510, USA}

\author{N.~Wuerfel\,\orcidlink{0000-0001-9872-5330}}
\affiliation{Randall Laboratory of Physics, University of Michigan, Ann Arbor, Michigan 48109, USA}

\author{Z.H.~Ye\,\orcidlink{0000-0002-1873-2344}}
\altaffiliation[Present address: ]{Department of Physics, Tsinghua University, Beijing 100084, China}
\affiliation{Physics Division, Argonne National Laboratory, Lemont, Illinois 60439, USA}

\collaboration{FNAL E906/SeaQuest Collaboration}


\date{\today}

\begin{abstract}
\end{abstract}


\maketitle

\section{Introduction}
\todo{are we comparing with E866 here?}
\todo{summarized previos finding?}
\todo{discribe the results from STAR and why this new result is better}

The results based on Run 2-3 have previously reported~\cite{dove2021,dove2023}. 
In this letter, we report the improved results using all data collected by the SeaQuest Experiment at Fermilab.
The data constitute a twofold increase in the detected muons compared to our previous publications.
\todo{also summarized any improvement in the analysis if any}

\section{The Drell-Yan Process}

\section{The SeaQuest Experiment}
\subsection{Target System}
\subsection{Spectrometer}
\subsection{Trigger System}

\section{Data Analysis}
\todo{progress between nature paper and now?}
\subsection{Track reconstruction}

\subsection{Monte Carlo simulation}
\todo{Explain the event generation, geat4 options, embeding(?), realization(?).}
\todo{How well is our data discribed by GMC? Which distributions are we comparing?}

\subsection{Mass-fit method}
\todo{Explain the method. If we are using NMSU mix, we will need to justify the normalization.}
\begin{figure}[htbp!]
	\centering
	\begin{subfigure}{\linewidth}
		\includegraphics[width=\linewidth]{massfit_run56_LH2.pdf}
	\end{subfigure}
	\begin{subfigure}{\linewidth}
		\includegraphics[width=\linewidth]{massfit_run56_LD2.pdf}
	\end{subfigure}
	\caption{Dimuon mass distribution for events collected
	on liquid hydrogen (top) and deuterium (bottom) targets for the second data set.
	The data points (solid squares) are compared with a fit (solid blue line) consisting of
	various components (see text).}
	\label{fig:massfit}
\end{figure}
\missingfigure[figwidth=\linewidth]{Drell-Yan yeild vs $x_T$ after background subtraction compared with MC} 
\todo{corrections(?): rate dependence, target contamination, live PoT}

\section{Measurement of The \texorpdfstring{$\sigma^{pd}/2\sigma^{pp}$}{pd/2pp} Drell-Yan Cross Section Ratio}
\todo{which variables do we want to show? (x2, x1, mass, pT, xF, etc)}
\begin{figure}[htbp!]
	\centering
	\includegraphics[width=\linewidth]{E906_E866_xT_CT18only.pdf}
	\caption{Measured $\sigma_{pd}/2\sigma_{pp}$ Drell-Yan cross section ratio from SeaQuest compared with E866/NuSea~\cite{towell2001}
		and calculations using CT18~\cite{hou2021} at different kinematics.}
	\label{fig:xT_csr}
\end{figure}

\todo{break down of systematics}
\todo{consistence of the data(source of systematics)}

\section{Extraction of \texorpdfstring{$\bar{d}\left(x\right)/\bar{u}\left(x\right)$}{dbar(x)/ubar(x)}
	and \texorpdfstring{$\bar{d}\left(x\right)-\bar{u}\left(x\right)$}{dbar(x)-ubar(x)}}
\todo{Explain the extraction. There are also addition systematics here.}
\todo{Where to put acceptance? We did show the acceptance in nature paper.}
\begin{figure}[htbp!]
	\centering
	\includegraphics[width=\linewidth]{E906_E866_dbarubar}
	\caption{Comparison of the extracted $\bar{d}\left(x\right)/\bar{u}\left(x\right)$ from SeaQuest
		compared with E866/NuSea~\cite{towell2001} and meson cloud~\cite{alberg2022} and statistical models~\cite{soffer2019}.}
	\label{fig:dbub_ratio}
\end{figure}
\begin{table*}[htbp!]
	\centering
	\caption{The measured $\sigma_{pd}/2\sigma_{pp}$ cross section ratio as well
		as the extracted $\bar{d}/\bar{u}$ and $\bar{d}-\bar{u}$ for each $x_{2}$ bin.
		The first uncertainty is statistical and the second systematic.}
	\begin{adjustbox}{max width=\textwidth}
		% !TeX root = ../main.tex
{
\renewcommand{\arraystretch}{1.5}
\begin{tabular}{cccccccc}
	\hline\hline
	$x_{2}$ range    & $\expval{x_{2}}$ & $\expval{x_{1}}$ & \begin{tabular}{@{}c@{}} $\expval{p_{T}}$\\(\unit{\GeV/c})\end{tabular} & \begin{tabular}{@{}c@{}}$\expval{M}$\\(\unit{\GeV/c^2})\end{tabular} & $\sigma_{pd}/2\sigma_{pp}$ & $\bar{d}/\bar{u}$                     & $\bar{d}-\bar{u}$                     \\ \hline
	$0.130$--$0.160$ & $0.146$          & $0.687$          & $0.760$                         & $4.71$                        & $1.177\pm0.033\pm0.028$    & $1.383^{+0.058+0.060}_{-0.053-0.060}$ & $0.176^{+0.021+0.024}_{-0.022-0.023}$ \\
	$0.160$--$0.195$ & $0.181$          & $0.610$          & $0.759$                         & $4.87$                        & $1.174\pm0.022\pm0.025$    & $1.431^{+0.041+0.061}_{-0.051-0.061}$ & $0.111^{+0.011+0.011}_{-0.011-0.011}$ \\
	$0.195$--$0.240$ & $0.222$          & $0.553$          & $0.760$                         & $5.11$                        & $1.275\pm0.024\pm0.027$    & $1.672^{+0.052+0.082}_{-0.052-0.082}$ & $0.092^{+0.012+0.012}_{-0.012-0.012}$ \\
	$0.240$--$0.290$ & $0.263$          & $0.516$          & $0.761$                         & $5.44$                        & $1.232\pm0.029\pm0.037$    & $1.653^{+0.073+0.123}_{-0.073-0.113}$ & $0.043^{+0.003+0.013}_{-0.003-0.013}$ \\
	$0.290$--$0.350$ & $0.324$          & $0.492$          & $0.762$                         & $5.83$                        & $1.205\pm0.040\pm0.052$    & $1.694^{+0.124+0.174}_{-0.114-0.174}$ & $0.024^{+0.004+0.004}_{-0.004-0.004}$ \\
	$0.350$--$0.450$ & $0.395$          & $0.474$          & $0.762$                         & $6.34$                        & $1.212\pm0.055\pm0.047$    & $1.925^{+0.205+0.205}_{-0.195-0.205}$ & $0.015^{+0.005+0.005}_{-0.005-0.005}$ \\ \hline\hline
\end{tabular}
}

	\end{adjustbox}
\end{table*}
\begin{table*}[htbp!]
	\centering
	\caption{Values of $\int_{0.45}^{0.13} \left[\bar{d}\left(x\right) - \bar{u}\left(x\right)\right] \dd{x}$
		and $\int_{0.45}^{0.13} x\left[\bar{d}\left(x\right) - \bar{u}\left(x\right)\right] \dd{x}$ at $Q^2=\SI{25.5}{\GeV\squared}$ extracted from
		SeaQuest compared with CT18, NNPDF4.0 PDFs as well as meson cloud and statistical models.}	
	\begin{adjustbox}{max width=\textwidth}
		% !TeX root = ../main.tex
\renewcommand{\arraystretch}{1.5}
\begin{tabular}{cccccccc}
\hline \hline
 &          & \multicolumn{3}{c}{PDFs} & & \multicolumn{2}{c}{Models} \\ \cline{3-5} \cline{7-8}
 & SeaQuest & CJ15 &CJ22     & NNPDF4.0     & & Stat.     & Meson cloud    \\ \hline
$\int^{0.45}_{0.13} \left[\bar{d}\left(x\right) - \bar{u}\left(x\right) \right]\dd{x}$ &
  $0.0179_{-0.0018}^{+0.0017} {}_{-0.0023}^{+0.0022}$ &
  $0.0153^{+0.0025}_{-0.0024}$ &
  $0.0167^{+0.0009}_{-0.0028}$ &
  $0.0208^{+0.0036}_{-0.0036}$ & &
  $0.0186$ &
  $0.0180$ \\
$\int^{0.45}_{0.13} x\left[\bar{d}\left(x\right) - \bar{u}\left(x\right) \right]\dd{x}$ &
  $0.00368_{-0.00036}^{+0.00034} {}_{-0.00049}^{+0.00045}$ &
  $0.00271^{+0.00050}_{-0.00046}$ &
  $0.00319^{+0.00019}_{-0.00063}$ &
  $0.00414^{+0.00078}_{-0.00078}$ & &
  $0.00386$ &
  $0.00361$ \\ \hline \hline
\end{tabular}

	\end{adjustbox}
\end{table*}
\section{Impact on PDF analysis}
\todo{Not sure about this part}

\section{Conclusions}

\begin{acknowledgments}
\todo{Add funding information}
\end{acknowledgments}
\bibliography{reference}

\end{document}

