\documentclass[letterpaper,12pt,oneside,final]{letter}
\usepackage[margin=1in]{geometry}
\usepackage{amsmath}
\usepackage{hyperref}
\usepackage{verbatim}
\usepackage{csquotes}
\usepackage{xcolor}
\usepackage{soul}
\usepackage{todonotes}
\setuptodonotes{inline}
\date{\today}
\makeatletter
\renewenvironment*{displayquote}
  {\begingroup\color{blue}\csq@getcargs{\csq@bdquote{}{}}}
  {\csq@edquote\endgroup}
\makeatother
\name{Ching Him Leung\\On behalf of the SeaQuest Collaboration}


\begin{document}
\begin{letter}{Physical Review Letters}
	\opening{Dear Dr.~Karthik,}
	We appreciate very much the positive assessment and constructive comments by the referees.
	As both referees recognize, the results presented in this paper are important for understanding the light sea flavor asymmetry of the proton.
	Below, we list the referees' comments (in blue), followed by our replies (in black).

	The revised manuscript and a marked copy, with all changes highlighted in blue, are attached.
	We have also modified some of the figures slightly to ensure the markers used are consistent
	throughout the paper. Some new references have been added,
	and sentences have been slightly modified for better clarity

	\hrulefill

	\textbf{Referee A}:
	\begin{displayquote}
		This is a very important result and should certainly be published.

		I have just a few comments:
	\end{displayquote}
	We thank Referee A for the careful reading and positive assessment of this work.
	We believe this paper provides crucial information for understanding the light quark sea.
	We address their comments in detail below.
	\begin{enumerate}
		\item \begin{displayquote}
			      It is stated that the transition from the parton model predictions of
			      Eq.~2 to the cross section is actually done at NLO.
			      Could this not have been done at NNLO?
			      Would it have made no difference?
		      \end{displayquote}
			  \begin{comment}
		      Thank you for the suggestion.
		      However, we do not believe an NNLO extraction would provide more useful information than the current NLO extraction.
		      The main purpose of the $\bar{d}/\bar{u}$ extraction performed in this paper is to illustrate how the measured
		      cross section ratio is related to the parton distributions, and the implication on future PDF analyses.
		      This extraction is performed with many simplifications, 
              such as the assumption of negligible nuclear effects in deuteron, 
              as mentioned in the manuscript.
		      An NNLO extraction would not be more informative given these simplifications.
		      PDF global analysis groups will use the measured cross section ratio,
			  which is the main result of this paper, in their global fits.
		      They are also better suited to handle the many theoretical uncertainties associated with the extraction.
			  \end{comment}
			  The SeaQuest collaboration does not currently have the capability to perform the NNLO extraction.
			  Although it is possible to develop such capability by collaborating with some theorists,
			  it would significantly delay the publication of this paper.
			  Once the final cross section ratio data from SeaQuest become available to the community,
			  we expect sophisticated analysis at the NNLO level will be performed by several PDF global analysis groups
		\item  \begin{displayquote}
			      On Fig.~1 in the key there is an entry called `mix'.
			      I think this is the background from accidental coincidence?
			      It would be helpful to say so.
		      \end{displayquote}
		      Yes, this refers to the background from accidental coincidence.
		      We have replaced the key in Fig.~1 with ``accidental'', which should clarify this point.
		\item \begin{displayquote}
			      Please can you tell us what `empty flask' means?
			      I know this is a letter and details are elsewhere,
			      but this is needed for comprehensibility
		      \end{displayquote}
		      Thank you for the suggestion. This is indeed an important point.
		      We have added a few sentences explaining `empty flask' to lines~213--216 in the updated manuscript.
		\item \begin{displayquote}
			      Again I know this is a letter but I am curious about the comparison of the two halves of the data,
			      which are combined. Could a figure not be put in `end matter'?
		      \end{displayquote}
		      Thank you for the suggestion. We agree that comparing the two halves of the data is important.
		      We have added a figure to the end matter as suggested, as well as a new Table listing the values.
		\item \begin{displayquote}
			      How are $x_1$ and $x_2$ defined in terms of experimental variables?
			      You only define $x_F$. I think these definitions are needed within the letter
		      \end{displayquote}
		      We agree that the definitions the variables $x_1$ and $x_2$ is necessary within the letter.
		      We have added the definition in lines~130--135 in the updated manuscript.
		\item \begin{displayquote}
			      In Fig.~3, lower right hand side, there seems to be no error band for the statistical model,
			      whereas there is on the left hand side. Is this a mistake? or is it invisible?
		      \end{displayquote}
		      Thank you for pointing this out.
			  We had reached out C.~Bourrely, one of the authors of Ref.~32.
              And he had provided us the updated predictions from statistical model prediction on $\bar{d}-\bar{u}$ and $\bar{d}/\bar{u}$.
			  We have updated both figures with the new predictions.
			  \\~\\
			  %It is not clear if we can get the error band
			  Thank you for pointing this out. We have contacted the authors of Ref.~32 to obtain the error band for the statistical model prediction on $\bar{d}-\bar{u}$, but we were unable to obtain the error bands for their 2015 prediction.
			  The error bands for the $\bar{d}/\bar{u}$ prediction were available in the original paper, but not the $\bar{d}-\bar{u}$ prediction.

	\end{enumerate}
    \begin{displayquote}
        If these few questions could be answered I would be very happy to recommend publication.
    \end{displayquote}
    We hope this will address the questions from Referee A and clarify the points raised.
    We appreciate their thoughtful comments and suggestions,
    which have helped improve the manuscript.
    
    \hrulefill

	\textbf{Referee B}:
	\begin{displayquote}
		This paper presents the final results from the SeaQuest experiment on the light-quark sea flavor asymmetry of the proton.
		While the behavior of this asymmetry at large $x$ has long been debated,
		these new measurements provide a definitive resolution.
		The results are also of significant interest because the observed asymmetry highlights
		nonperturbative contributions that go beyond the standard perturbative QCD framework.

		This work expands on the previous SeaQuest analyses reported in Refs.~23 and 24 by using the complete dataset,
		which contains roughly twice as many detected muons, and by applying comparable analysis methods.
		The results are presented in detail, enabling direct comparisons with both the earlier
		SeaQuest results and the E866 NuSea asymmetry measurements.

		This work is clearly presented and suitable for publication in its present form.
		I have no suggested changes to the text.

		This study merits publication in Physical Review Letters,
		as it provides a definitive resolution to longstanding questions regarding flavor asymmetry.
		The measurement offers significant and unique insight into the behavior of the sea-quark asymmetry at large $x$.
		In particular, the results demonstrate a $\bar{d}/\bar{u}$ ratio exceeding unity at high $x$,
		with substantially reduced uncertainties relative to previous measurements—an issue that had remained unresolved prior to this work.
		These results will be incorporated into future parton distribution function fits,
		which are essential inputs for a wide range of measurements involving protons and nuclei.
	\end{displayquote}
	We thank Referee B for recognizing the significance of this work and their positive assessment.
	We agree that the results presented in this paper provide important constraints on the behavior of the light sea flavor asymmetry,
	which will be used in many global analyses of parton distribution functions in the foreseeable future.

    \hrulefill
	\closing{Sincerely,}
\end{letter}
\end{document}