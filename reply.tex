\documentclass[letterpaper,12pt,oneside,final]{letter}
\usepackage[margin=1in]{geometry}
\usepackage{amsmath}
\usepackage{hyperref}
\usepackage{comment}
\usepackage{csquotes}
\usepackage{xcolor}
\usepackage{soul}
\usepackage{todonotes}
\setuptodonotes{inline}
\date{\today}
\makeatletter
\renewenvironment*{displayquote}
  {\begingroup\color{blue}\csq@getcargs{\csq@bdquote{}{}}}
  {\csq@edquote\endgroup}
\makeatother
\name{Ching Him Leung\\On behalf of the SeaQuest Collaboration}


\begin{document}
\begin{letter}{Physical Review Letters}
	\opening{Dear Dr.~Karthik:}
    We appreciate very much the positive assessment and constructive comments by the referees.
    The results presented in this paper are important for understanding the light sea flavor asymmetry,
    which both referees also agree upon.
    Below we list the referees' comments (in blue), followed by our replies (in black).
    The revised manuscript and a marked copy, with all the changes highlighted in blue, are attached. 

    \textbf{Referee A}:
    \begin{displayquote}
        This is a very important result and should certainly be published.

        I have just a few comments:
    \end{displayquote}
    We thank Referee A for the careful reading and positive assessment of this work.
    We believe this paper provides useful information on the light quark sea.
    We address their comments in detail below.
    \begin{enumerate}
        \item \begin{displayquote}
            It is stated that the transition from the parton model predictions of
            Eq.~2 to the cross section is actually done at NLO.
            Could this not have been done at NNLO?
            Would it have made no difference?
        \end{displayquote}
        Thank you for the suggestion. 
        However we do not believe a NNLO extraction would provide any more useful information than the current NLO extraction.
        The main purpose of the $\bar{d}/\bar{u}$ extraction performed in this paper is to illustrate how the measured
        cross section ratio is related to the parton distributions, and the implication on future PDF analyses. 
        This extraction is performed with many simplification,
        such as the assumption of negligible nuclear effects in deuteron as mentioned in the manuscript.
        And therefore this extraction is not as accurate as a proper global global analysis.
        A NNLO extraction would not be any more informative given these simplification.
        PDF global analysis groups will use the measured cross section ratio, which is the main result of this paper,
        in their global fits. 
        And they are also better suited to handle the theoretical uncertainties associated with the extraction. 
        \item  \begin{displayquote}
            On Fig.~1 in the key there is an entry called `mix'. 
            I think this is the background from accidental coincidence? 
            It would be helpful to say so.
        \end{displayquote}
        Yes. It is the background from accidental coincidence.
        We have replaced the key in Fig.~1 to ``accidental'', this should clarify this point.
        \item \begin{displayquote}
            Please can you tell us what `empty flask' means?
            I know this is a letter and details are elsewhere,
            but this is needed for comprehensibility
        \end{displayquote}
        Thank you for the suggestion. This is indeed and important point. 
        We have added a sentence `empty flask' to line~?? in the updated manuscript.
        \item \begin{displayquote}
            Again I know this is a letter but I am curious about the comparison of the two halves of the data,
            which are combined. Could a figure not be put in `end matter'?
        \end{displayquote}
        Thank you for the suggestion. We agree that the comparison between the two halves of the data would be important. 
        We have added a figure to the end matter as Referee A has suggested. 
        \todo{Show the CSR vs $x_2$ in the end matter. Show the updated Run 2-3, Run 5-6, and combined results.}
        \item \begin{displayquote}
            How are $x_1$ and $x_2$ defined in terms of experimental variables?
            You only define $x_F$. I think these definitions are needed within the letter
        \end{displayquote}
        We agree we should have defined the variables $x_1$ and $x_2$ in the original manuscript.
        We have added the definition in line~??? in the updated manuscript
        \todo{Add the definition to main text or end matter?}
        \item \begin{displayquote}
            In Fig.~3, lower right hand side, there seems to be no error band for the statistical model,
            whereas there is on the left hand side. Is this a mistake? or is it invisible?
        \end{displayquote}
        %If we can get the error bands
        Thank you for pointing this out. With help from C.~Bourrley, one of the author of Ref.~?,
        we have added the error bands to the lower left panel.
        \todo{Not sure what the response should be.
        I don't have the error band in the Fortran code the author sent us.  }
        %If we can NOT get the error bands
        Thank you for point this out. The error bands on the lower right panel is obtained by ???
        We have reached out to the authors of Ref.~?, and they have send us the central value of their predictions.
        We have reached out to them again, but are unable to get this error bands at this time.
    \end{enumerate}

    \textbf{Referee B}:
    \begin{displayquote}
        This paper presents the final results from the SeaQuest experiment on the light-quark sea flavor asymmetry of the proton. 
        While the behavior of this asymmetry at large $x$ has long been debated,
        these new measurements provide a definitive resolution.
        The results are also of significant interest because the observed asymmetry highlights
        nonperturbative contributions that go beyond the standard perturbative QCD framework.

        This work expands on the previous SeaQuest analyses reported in Refs.~23 and 24 by using the complete dataset,
        which contains roughly twice as many detected muons, and by applying comparable analysis methods.
        The results are presented in detail, enabling direct comparisons with both the earlier
        SeaQuest results and the E866 NuSea asymmetry measurements.

        This work is clearly presented and suitable for publication in its present form.
        I have no suggested changes to the text.

        This study merits publication in Physical Review Letters, 
        as it provides a definitive resolution to longstanding questions regarding flavor asymmetry.
        The measurement offers significant and unique insight into the behavior of the sea-quark asymmetry at large $x$.
        In particular, the results demonstrate a $\bar{d}/\bar{u}$ ratio exceeding unity at high $x$,
        with substantially reduced uncertainties relative to previous measurements—an issue that had remained unresolved prior to this work. 
        These results will be incorporated into future parton distribution function fits,
        which are essential inputs for a wide range of measurements involving protons and nuclei.
    \end{displayquote}
    We thank Referee B for the recognizing the significance of this work and their positive assessment.
    We agree that the results presented in this paper provide important constraints on the behavior of the light sea flavor asymmetry, 
    which will be incorporated into many global analyses of parton distribution functions in the foreseeable future.

    \closing{Sincerely,}
\end{letter}
\end{document}