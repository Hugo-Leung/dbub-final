\documentclass{article}
\usepackage[margin=1in]{geometry}
\usepackage{amsmath}
\usepackage{siunitx}
\usepackage{physics}
\usepackage{hyperref}
\usepackage{cleveref}
\usepackage[style=phys,%
	biblabel=brackets,%
	chaptertitle=false,pageranges=false,%
maxnames=4,sorting=none,eprint]{biblatex}
\addbibresource{ref.bib}
\usepackage{authblk}
\usepackage{subcaption}

\title{Outline of the Final $\bar{d}/\bar{u}$ Paper}
\author{Ching Him Leung}
\date{\today}
\begin{document}
\maketitle
Title: ``Improve measurement of flavor asymmetry of the light-quark sea in the proton with Drell-Yan production in
	\texorpdfstring{$p+p$}{p+p} and \texorpdfstring{$p+d$}{p+d} collisions at \texorpdfstring{\SI{120}{\GeV}}{120~GeV}''
\begin{enumerate}
    \item Section: Introduction
    \begin{enumerate}
        \item summarize the improvements as compared to the previous paper
        \begin{enumerate}
            \item double the size of data, hence reduced statistical uncertainties.
            \item different trigger roadset, hence can be to used to check for potential unknown sources of systematics.
            \item Improved analysis techniques.
        \end{enumerate}
        \item perhaps state that results on the $J/\psi$ production has been reported in [ref] in the introduction.
    \end{enumerate} 
    \item Section: Drell-Yan Process
    \begin{enumerate}
        \item Define the kinematics variables $x_1$, $x_2$, and $x_F$
        \item Explain how the $\sigma_{pd}/2\sigma_{pp}$ ratio is related the $\bar{d}/\bar{u}$
    \end{enumerate}
    \item Section: The SeaQuest Experiment
    \begin{enumerate}
        \item Target System
        \item Spectrometer and Tracking Stations
        \item Trigger System. Explain FPGA1, FPGA4, and NIM3, and their purpose
    \end{enumerate}
    \item Section: Data Analysis
    \begin{enumerate}
        \item Monte Carlo simulation
        \begin{itemize}
            \item Event generation
            \item GEANT4 settings
            \item realization
            \item NIM3 embedding
            \item Data/GMC comparisons (Or should we put this after discussing massfit?)
            \item We might want to use the new reweighted GMC from Dinupa. Need to investigate more.
                However, I am not keen on explaining the ML driven reweighting.
                I might be a bit old-school, I prefer to frame reweighting as fine tuning the input parameters (eg.~$p_T,\,\bar{d}/\bar{u}$, etc).
        \end{itemize}
        \item Track reconstruction
        \begin{itemize}
            \item some comment on the event selection
            \begin{itemize}
                \item We use loose cuts during the fit, then impose tighter cuts during projection.
                \item additional $x_B$ and mass cut during projection
            \end{itemize}
        \end{itemize}
        \item massfit
        \begin{itemize}
            \item explain the track mixing method(s) (FPGA4mix and/or FPGA1mix), we would then quote the differences as systematics.
            \item Corrections:
            \begin{itemize}
                \item live PoT
                \item rate dependence
                \item target contamination (?)\\
                    We might want to admit our previous mistakes here.
            \end{itemize}
            \item figure:
            \begin{itemize}
                \item mass fit for LH2 and LD2
                \item background subtracted yield compared with GMC. Which variable(s) should we show?
            \end{itemize}
        \end{itemize}
    \end{enumerate}
    \item Section: Measurement of The \texorpdfstring{$\sigma_{pd}/2\sigma_{pp}$}{pd/2pp} Drell-Yan Cross Section Ratio
    \begin{enumerate}
        \item figure:
        \begin{itemize}
            \item $\sigma_{pd}/2\sigma_{pp}$ vs $x_2$
            \begin{itemize}
                \item Comparing results from long paper and combined results.\\
                ``the results from Ref.~[ref]  are compared with the results obtained with the improved analysis and full dataset''
                \item Comparing SeaQuest and NuSea.
                \item Which PDFs are we planning to show? CT18, NNPDF4.0, or others?
                    It might be interesting to compare the results from the same group before and after nature paper.
                    Namely, NNPDF3.0 vs NNPDF4.0, or CJ15 vs CJ22. 
            \end{itemize}
            \item Not sure about other variables at this point. I think mass and $p_T$ would be interesting.
            But I am not sure what new information $x_1$ or $x_F$ would bring, aside from comparing with previous results. 
            \begin{itemize}
                \item $x_F$ might be useful for comparing with $J/\psi$.
                \item $x_1$ would be useful for nuclear dependence studies. But we don't have any results yet.
            \end{itemize}
        \end{itemize}
        \item tables:
        \begin{itemize}
            \item The measured $\sigma_{pd}/2\sigma_{pp}$ cross section ratio as well
		as the extracted $\bar{d}/\bar{u}$ and $\bar{d}-\bar{u}$ for each $x_{2}$ bin.
            \item Breakdown of systematic uncertainty in the cross section ratio
        \end{itemize}
        \item comment on the consistency between datasets
        \item comment of the different beam energies between SeaQuest and NuSea, therefore expected differences in the cross section ratio
    \end{enumerate}
    \item Section: Extraction of \texorpdfstring{$\bar{d}\left(x\right)/\bar{u}\left(x\right)$}{dbar(x)/ubar(x)}
	and \texorpdfstring{$\bar{d}\left(x\right)-\bar{u}\left(x\right)$}{dbar(x)-ubar(x)}
    \begin{enumerate}
        \item Explain the extraction, and the additional systematics
        \item Should we also include the acceptance table here?
        \item figures:
        \begin{itemize}
            \item $\bar{d}/\bar{u}$
            \begin{itemize}
                \item comparisons with E866
                \item comparisons with meson cloud and statistical models
                \item comparisons with PDFs
                \item comparisons with long paper? I think this is less important.
            \end{itemize}
            \item $\bar{d}-\bar{u}$
            \begin{itemize}
                \item also include SIDIS data from HERMES
            \end{itemize}
        \end{itemize}
        \item tables:
        \begin{itemize}
            \item $\int_{0.45}^{0.13} \left[\bar{d}\left(x\right) - \bar{u}\left(x\right)\right] \dd{x}$
            \item $\int_{0.45}^{0.13} x\left[\bar{d}\left(x\right) - \bar{u}\left(x\right)\right] \dd{x}$
            \item statistical model and meson cloud model
            \item PDFs
        \end{itemize}
    \item Section: Absolute cross section from full dataset
    \begin{enumerate}
        \item Figure:
        \begin{itemize}
            \item All $x_F$ bin in a single plot, one per target. 
        \end{itemize} 

    \end{enumerate}
    \end{enumerate}
\end{enumerate}
\end{document}

