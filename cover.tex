\documentclass[letterpaper,12pt,oneside,final]{letter}
\usepackage[margin=1in]{geometry}
\usepackage{amsmath}
\usepackage{hyperref}

\name{Ching Him Leung\\On behalf of the SeaQuest Collaboration}


\begin{document}
\begin{letter}{Physical Review Letters}
	\opening{Dear Editors:}
	We are submitting a manuscript entitled ``Final SeaQuest results on the flavor asymmetry of the proton
	light-quark sea with proton-induced Drell-Yan process" for consideration by Physical Review Letters.

	We report the updated SeaQuest measurements of the Drell-Yan $\sigma_{\textrm{pd}}/2\sigma_{\textrm{pp}}$ cross section ratio
	from the analysis of the full dataset. Compared with our previous publications,
	the combined dataset yields roughly twice the number of detected muons.
	We find that the Drell-Yan $\sigma_{\textrm{pd}}/2\sigma_{\textrm{pp}}$ ratio is consistently above unity,
	and that the $\bar{d}(x)/\bar{u}(x)$ ratio increases with $x$ over the range $0.13<x<0.45$.
	These results reinforce earlier indications of an excess of $\bar d$ over $\bar u$ and
	are in good agreement with several theoretical models, including  statistical and meson-cloud models.

	This work will be of interest to a broad audience.
	In particular, it provides important constraints for global parton-distribution-function (PDF) analyses,
	since it is one of the few measurements that is able to constrain the light sea-quark flavor asymmetry at large $x$
	in the foreseeable future. The previously published SeaQuest results reduced uncertainties on the flavor asymmetry;
	the final results reported here will further constrain antiquark distributions in future global analyses.
	Improved knowledge of the nucleon partonic structure at large $x$ is also important for searches for
	beyond-the-Standard-Model (BSM) physics at the LHC, because PDFs enter many theoretical predictions.
	With recent developments in lattice QCD using large-momentum effective theory,
	these results will provide a valuable benchmark for future lattice calculations.

	We would like to suggest the following potential referees, who are experts on this subject:
	\begin{itemize}
		\item Prof.~Haiyan Gao, Duke University (\href{mailto:haiyan.gao@duke.edu}{haiyan.gao@duke.edu})
		%\item Dr.~Tim Hobbs, Argonne National Laboratory (\href{mailto:tim@anl.gov}{tim@anl.gov})
		\item Dr.~Cynthia Keppel, Thomas Jefferson National Accelerator Facility (\href{mailto:keppel@jlab.org}{keppel@jlab.org})
		\item Prof.~Gerald Miller, University of Washington (\href{mailto:miller@uw.edu}{miller@uw.edu})
		\item Prof.~Richard Milner, Massachusetts Institute of Technology (\href{mailto:milner@mit.edu}{milner@mit.edu})
		\item Prof.~Pavel Nadolsky, Michigan State University (\href{mailto:nadolsky@msu.edu}{nadolsky@msu.edu})
		\item Prof.~Fredrick Olness, Southern Methodist University (\href{mailto:olness@smu.edu}{olness@smu.edu})
		\item Prof.~Bernd Surrow, Temple University (\href{mailto:surrow@temple.edu}{surrow@temple.edu})
		%\item Dr.~Ramona Vogt, Lawrence Livermore National Laboratory (\href{mailto:vogt2@llnl.gov}{vogt2@llnl.gov})
	\end{itemize}

	We confirm that this manuscript has not been published elsewhere and is not under consideration by another journal.
	All authors have approved the manuscript and agree with its submission to Physical Review Letters.
	Thank you for your consideration.
	\closing{Sincerely,}
\end{letter}
\newpage

Justification paragraph (no more than 100 words)\\~\\
This paper reports the SeaQuest measurements of the Drell-Yan $\sigma_{\textrm{pd}}/2\sigma_{\textrm{pp}}$
cross-section ratio from the analysis of the full dataset
and shows that the $\bar{d}(x)/\bar{u}(x)$ ratio increases with $x$ in $0.13< x<0.45$,
reinforcing earlier indications of a $\bar{d}$ excess.
The result provides crucial constraints for global PDF analyses — one of the few measurements probing the
light sea-quark asymmetry at large $x$ in the near term — and improves knowledge of nucleon structure
used in many LHC predictions. It will also serve as a benchmark for future lattice-QCD calculations.

\end{document}
